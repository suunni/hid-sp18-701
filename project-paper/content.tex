% status: 0
% chapter: Application


\def\paperstatus{0} % a number from 0-100 indicating your status. 100
                % means completed
\def\paperchapter{Aplication} % This section is typically a single keyword. from
                   % a small list. Consult with theinstructors about
                   % yours. They typically fill it out once your first
                   % text has been reviewed.
\def\hid{hid-sp18-701} % all hids of the authors of this
                                % paper. The paper must only be in one
                                % authors directory and all other
                                % authors contribute to it in that
                                % directory. That authors hid must be
                                % listed first
\def\volume{9} % the volume of the proceedings in which this paper is to
           % be included

\def\locator{\hid, Volume: \volume, Chapter: \paperchapter, Status: \paperstatus. \newline}

\title{Detection of street signs in videos in a robot swarm}


\author{Sunanda Unni}
\orcid{1234-5678-9012}
\affiliation{%
  \institution{Indiana University}
  \streetaddress{Smith Research Center}
  \city{Bloomington} 
  \state{IN} 
  \postcode{47408}
  \country{USA}
}
\email{suunni@indiana.edu}

\author{Gregor von Laszewski}
\affiliation{%
  \institution{Indiana University}
  \streetaddress{Smith Research Center}
  \city{Bloomington} 
  \state{IN} 
  \postcode{47408}
  \country{USA}}
\email{laszewski@gmail.com}


% The default list of authors is too long for headers}
\renewcommand{\shortauthors}{G. v. Laszewski}


\begin{abstract}
  Extracting and identifying traffic signals from the videos captured
  by Robot swarms to help in recognizing the pattern and benchmarking
  the performance of the setup using cloudmesh.
\end{abstract}

\keywords{\locator\ TensorFlow, python, ImageNet, Cloudmesh}


\maketitle

\section{Steps}
Create the model for identifying the red, yellow and green signals
using python and opencv\cite{www-opencv}.  Create cloudmesh command
extension for identifying the signal in a given picture.


\begin{acks}

  The authors would like to thank Dr.~Gregor~von~Laszewski for his
  support and suggestions to write this paper.

\end{acks}

\bibliographystyle{ACM-Reference-Format}
\bibliography{report} 

